% !TEX TS-program = pdflatexmk  
\pdfcompresslevel=0
\pdfobjcompresslevel=0
\documentclass[11pt]{amsart}
\pdfgentounicode=1 
\input glyphtounicode.tex 
\InputIfFileExists{glyphtounicode-cmr.tex}{}{} 
\InputIfFileExists{glyphtounicode-ntx.tex}{}{}
\pdfglyphtounicode{afii61664}{200B}
\usepackage[margin=1.5in]{geometry} 
\usepackage[parfill]{parskip}% Begin paragraphs with an empty line rather than an indent
\usepackage{graphicx}
\pdfmapfile{=erewhon.map}
%SetFonts
%\pdfmapline{+erewMI erewMI <erewMI.pfb}
%\pdfmapline{+erewBMI erewBMI <erewBMI.pfb}
%\pdfmapline{+erewMR erewMR <erewMR.pfb}
%\pdfmapline{+erewBRM erewBRM <erewBRM.pfb}
\usepackage[a-1b]{pdfx}
% erewhon+newtxmath
\usepackage[sups,p,osf,scaled=.98,space]{erewhon}
\usepackage[T2A,OT2,T1]{fontenc}
\usepackage{textcomp}
\usepackage{cabin}
\usepackage[varqu,varl]{zi4}% inconsolata
\usepackage[utopia,vvarbb]{newtxmath}
%SetFonts
\usepackage{fonttable}
\title{LaTeX Support for Erewhon}
\author{Michael Sharpe}
\date{\today}  % Activate to display a given date or no date
\begin{document}
\maketitle
%\textbf{\textsc{Test}}

\emph{Erewhon} is a font package based largely on Andrey V.\ Panov's \emph{Heuristica}, but with so many changes that it is no longer strictly compatible with that package, and is offered instead as an enhanced alternative. (\emph{Heuristica}  extended the \emph{Utopia} font family made available by the \TeX\ Users' Group, adding many accented glyphs, Cyrillic glyphs, ligatures, superior and oldstyle fixed-width figures in all styles, and Small Caps in Regular style only. It is widely distributed as a free font collection in OpenType, TrueType and Type$1$ formats.) \emph{Erewhon} is provided in OpenType and Type$1$ formats with \LaTeX\ support files in encodings T$1$, TS$1$, LY$1$, OT$2$, T$2$A, T$2$B and T$2$C.  Changes made in the transition from \emph{Heuristica} to \emph{Erewhon} include:
\begin{itemize}
\item
\textsl{slanted} as well as \textit{Italic} shapes;
\item
\textsc{Small Caps} in \textsc{\textbf{bold}} as well as \textsc{regular} upright shapes, with \textsc{\textit{italic}} and \textsc{\textsl{slanted  Small Caps}} shapes from the slanted variants;
\item expanded lookup tables in the {\tt.otf} files for users of XeLaTeX and LuaLaTeX;
\item a number of f-ligatures have been modified, and a \verb|T_h| ligature added;
\item proportionally spaced figures (lining and oldstyle), adding to the existing taboldstyle figures;
\item full collections of superior lowercase letters (including \`e as \textsu{\`e} and \'e as \textsu{\'e}), mainly for the benefit of languages in which those are in common use---e.g., French, Spanish;
\item size reduced by 6\% from Heuristica, which matched the old version of Utopia---the new size matches that of Adobe's commercial UtopiaStd;
\item shapes of some oldstyle figures modified to have more of an oldstyle appearance;
\item fraction macros based on the new numerator and denominator figures: e.g., \verb|\textfrac{34}{73}| renders as \textfrac{34}{73} and \verb|\textfrac[2]{17}{43}| renders as \textfrac[2]{17}{43};
\item the bold upright face has been made less cramped.
\item As of version $1.109$, a Russian ruble glyph from Heuristica was added but is not accessible from LaTeX encodings. For LaTeX users, a \verb|\textruble| command has been added, built as a composite glyph built from a scaled down P and two rules, providing an approximation to the actual ruble glyph: \textruble.
\end{itemize}

The {\tt newtx} package has been modified, as of version $1.26$,  to offer a new option {\tt utopia} (or, equivalently, {\tt heuristica} or {\tt erewhon}) that uses math italic glyphs taken from Utopia and oldstyle figures from \emph{Erewhon}. Its slanted Greek alphabets are constructed from the {\tt txfonts} slanted Greek letters by reducing their italic angle from $15.5$\textdegree\ to $13$\textdegree, matching Utopia's italic angle. 

\textbf{Important Note:} Starting with version 1.10 (2019-11-04), the mathematical Greek glyphs are derived mainly from those in the {\tt fourier} package.

For Erewhon text and matching math, you can use\footnote{There is most likely also a way to use {\tt MathDesign} or {\tt fourier} with at least partial compatibility.}:
\begin{verbatim}
% load babel package and options before erewhon
\usepackage[p,osf,scaled=.98,space]{erewhon} % scaling by .98 not really necessary
\usepackage[varqu,varl]{inconsolata} % typewriter
\usepackage[type1,scaled=.95]{cabin} % sans serif like Gill Sans
\usepackage[utopia,vvarbb]{newtxmath}
\end{verbatim}
The effect of the options {\tt p,osf} is to force the default figure style in {\tt erewhon} text to be proportional oldstyle 0123456789 while using lining figures $0123456789$ in math mode. If no options are specified, tabular lining figures will be used throughout.

With the settings above, there is now almost no difference between its output and the output from the {\tt fourier} package, except:
\begin{itemize}
\item
the math axis using {\tt fourier} will be about {\tt .03pt} higher;
\item rules and math symbols in {\tt fourier} will be about \verb|5%| less thick.
\end{itemize}

\textsc{Options available:}
\begin{itemize}
\item The option {\tt scaled} allows you to change the scale. E.g., if you want \emph{Erewhon} to render at the same size as the original \emph{Utopia} or \emph{Heuristica}, use {\tt scaled=1.064}.
\item
The option {\tt proportional}, or, equivalently, {\tt p}, specifies the use of proportional rather than the default tabular figures.
\item
 The {\tt space} option allows you to specify a factor by which to increase the interword spacing, which is, IMO, a bit tight.
\item
The option {\tt oldstyle}, or, equivalently, {\tt osf}, specifies oldstyle figures in text mode---math mode always uses tabular lining figures. By itself, {\tt osf} results in tabular oldstyle figures unless you also specify the option {\tt p}, or {\tt proportional}.
\item The option {\tt scosf} changes the figure style to {\tt osf} only within small caps. 
\item
 The option {\tt sups}  changes the footnote marker style to use the superior figures from \emph{Erewhon} rather than the default  superscripts based on reduced lining figures, which usually appear too light. (The {\tt superiors} package offers further options.)
\end{itemize}

\emph{Erewhon} is so austere for a text font and \emph{Inconsolata} is so fancy for a typewriter font that you may find they blend together all too well. For more of a distinction replace the {\tt inconsolata} line above with
\begin{verbatim}
\usepackage{zlmtt} % serifed typewriter font extending cmtt
\end{verbatim}

As Utopia text is rather cramped, you might try  applying a small amount of letterspacing (tracking) and increasing the interword spacing by means of the {\tt microtype} package. As of version $1.08$, this provides a number of ways to modify interword spacing by specifying one or more of the options described below. Erewhon  word-spacing is governed by three quantities: 
\begin{itemize}
\item
{\tt spacing} (default value {\tt .211em}), \verb|\fontdimen2| of the main text font.
\item
{\tt stretch} (default value {\tt 1.055em}), \verb|\fontdimen3| of the main text font.
\item
{\tt shrink} (default value {\tt .0703em}), \verb|\fontdimen4| of the main text font.
\end{itemize}
(Note the use of {\tt em} values rather than absolute values so that word spacing responds to scale changes.) You may  modify these values individually or by setting a value for the option {\tt space} or by specifying {\tt looser} or {\tt loosest}.
\begin{itemize}
\item
Option {\tt spacing=.24em} would change the spacing from {\tt.211em} to {\tt.24em}.
\item
Option {\tt stretch=.14em} would change the stretch from {\tt.1055em} to {\tt.14em}.
\item
Option {\tt shrink=.1em} would change the shrink from {\tt.0703em} to {\tt.1em}.
\item
Option {\tt space=1.2} would multiply each word-spacing parameter by the factor {\tt1.2}. (The option {\tt space} with no value would result in a factor of {\tt 1.23}, leading to a spacing value of close to {\tt.26em}. This documentation uses option {\tt space} with no value specified.)
\item
Option {\tt looser} would change the three parameters to {\tt.25em, .125em,.1em} respectively. 
\item
Option {\tt loosest} would change the three parameters to {\tt.28em, .125em,.13em} respectively. 
\end{itemize}

\textsc{Supplementary figures and alphabets:}
Beginning with version 1.107, the {\tt superiors} and {\tt numerators} have been substantially revised along the following lines:
\begin{itemize}
\item
The baseline for {\tt superior} letters and figures has been uniformized at 356 {\tt units}. (Recall that, like most TeX fonts, 100 units is equal to 1{\tt bp} for 10{\tt pt} typesetting.)
\item
The baseline for {\tt numerator} figures has been uniformized at 256 {\tt units}.
\item A substantial number of {\tt superior} letters and symbols have been added---see the table near the end of this document.
\item 
The TS$1$ encoded ({\tt textcomp}) coverage has been enriched and is now essentially full. In particular, this permits macros like \verb|\textfractionsolidus| and \verb|\textcircled| render using {\tt erewhon} glyphs rather than Computer Modern glyphs. (Recall that if \LaTeX\ thinks the TS$1$ font lacks enough glyphs, it tries to make substitutions first from the OMS font, then from Computer Modern. As {\tt newtxmath} lacks an OMS encoded font, the substitutions are made only from Computer Modern.) 
\end{itemize}


\textsc{Macros:}
\begin{itemize}
\item
\verb|\textlf| and \verb|\texttlf| render their arguments in proportional and tabular lining figures respectively, no matter what the default figure style. E.g., \verb|\textlf{345}| produces \textlf{345}.
\item
\verb|\textosf| and \verb|\texttosf| render their arguments in proportional and tabular oldstyle figures respectively, no matter what the default figure style. For example, \verb|\textosf{345}| produces~\textosf{345}.
\item
\verb|\textsu|  renders its argument in superior figures, no matter what the default figure style. E.g., \verb|\textsu{345}| produces \textsu{345}.
\item
\verb|\textin|  renders its argument in inferior figures, no matter what the default figure style. E.g., \verb|\textin{345}| produces \textin{345}.
\item
\verb|\textnu|  renders its argument in numerator figures, no matter what the default figure style. E.g., \verb|\textnu{345}| produces \textnu{345}.
\item
\verb|\textde|  renders its argument in denominator figures, no matter what the default figure style. E.g., \verb|\textde{345}| produces \textde{345}.
\item
\verb|\textfrac|  renders its two arguments as a vulgar fraction, using \verb|\textnu| for the numerator and \verb|\textde| for the denominator. E.g., \verb|\textfrac{31}{64}| produces~\textfrac{31}{64}.
\item
\verb|\textcircled|  renders its argument as a circled small uppercase letter. E.g., \verb|\textcircled{q}| produces~\textcircled{q}.
\end{itemize}
\textsc{Very Brief, Nonsensical Math Example:}\\
Let $B(X)$ be the set of blocks of $\Lambda_{X}$
and let $b(X) \coloneq |{B(X)}|$ so that $\hat\phi=\sum_{Y\subset X}(-1)^{b(Y)}b(Y)$. 

\textsc{Enhanced Mathematical Options:}\\
As of version {\tt 1.10}, the {\tt utopia} option to {\tt newtxmath} enjoys enhanced status. Most importantly, this means that you can generate pdf output satisfying the PDF/A-1b standards. It means, in addition, that the option {\tt subscriptcorrection} is available, permitting you to improve the spacing of subscripts by copying {\tt newtx-erewhon-subs.tex} to your home texmf tree and modifying its entries. For details, consult Appendix 2 of {\tt newtxdoc.pdf}, the documentation file for the {\tt newtx} package.

\textsc{Unicode TeX processing:}\\
There is a file in the distribution named {\tt erewhon.fontspec} spelling out the file names of the Erewhon Opentype fonts, so for using Erewhon text, it is enough to write
\begin{verbatim}
\usepackage[no-math]{fontspec}
\setmainfont{erewhon}
\end{verbatim}
There are two stylistic sets available:\\
\begin{itemize}
\item
{\tt ss01} substitutes {\tt one.[tab]oldstyle.alt} for {\tt one.[tab]oldstyle}, the {\tt .alt} form having the appearance of a shortened I.
\item
{\tt ss02} changes the behavior of typesetting {\tt uni1E9E}, German capital sharp s. The default is to use the new symbol. To change to the old for using SS and the like, set {\tt ss02}.
\item
{\tt ss03} changes the rendering of the Euro symbol from Erewhon's \texteuro\ to the official (sans) version.
\end{itemize} 



\textsc{Greek symbols in the opentype fonts:}\\
As of version $1.103$ ($2020-03-19$) the opentype text fonts contain the Greek symbols from the Fourier package, and also the symbols {\tt Ohm} and {\tt Ohminv}. This may be a temporary measure that will disappear after {\tt fontspec} is able to fully mimic the capabilities of virtual fonts.
\newpage
\textsc{Glyph Coverage Examples}\\
\textbf{Erewhon-Regular-tlf-t1}\\
\fonttable{Erewhon-Regular-tlf-t1}
\newpage
\textbf{Erewhon-Regular-tlf-sc-t1}\\
\fonttable{Erewhon-Regular-tlf-sc-t1}
\newpage
\textbf{Erewhon-Regular-tlf-ly1}\\
\fonttable{Erewhon-Regular-tlf-ly1}
\newpage
\textbf{Erewhon-Regular-tlf-ts1}\\
\fonttable{Erewhon-Regular-tlf-ts1}
\newpage
\textbf{Superior letters and figures: Erewhon-Regular-sup-t1}\\
\fonttable{Erewhon-Regular-sup-t1}
\newpage

\textbf{Erewhon-Regular-tlf-ot2}\\


\fonttable{Erewhon-Regular-tlf-ot2}

(This 7-bit encoding is intended for users lacking a Cyrillic keyboard. For further information, consult the documentation for the package {\tt nimbus15}.)
\newpage
\textbf{Erewhon-Regular-tlf-t2a}\\
\fonttable{Erewhon-Regular-tlf-t2a}
\newpage
\textbf{Erewhon-Regular-tlf-t2b}\\
\fonttable{Erewhon-Regular-tlf-t2b}
\newpage
\textbf{Erewhon-Regular-tlf-t2c}\\
\fonttable{Erewhon-Regular-tlf-t2c}
\newpage
\textbf{Erewhon-Regular-tlf-sc-t2a}\\
\fonttable{Erewhon-Regular-tlf-sc-t2a}


\end{document}   